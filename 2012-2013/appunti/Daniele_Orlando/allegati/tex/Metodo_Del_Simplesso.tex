\documentclass{article}

\usepackage[active,tightpage]{preview}
\setlength\PreviewBorder{2pt}
\usepackage[paperwidth=21cm]{geometry}

\usepackage{mathtools}
\usepackage{breqn}
\usepackage{setspace}

\begin{document}
        \begin{preview}
                \[
                    min -5x_1 -x_2 -x_3 + x_4
                \]
                \paragraph{}
                \begin{math}
                        \begin{dcases}
                                \begin{tabular}{r r r r l}
                                        $ x_1 $   &           & $ + 2 x_3 $ & $ - 2 x_4 $ & $ = 2 $ \\
                                        $ 3 x_1 $ & $ + x_2 $ & $ - x_3 $   & $ + x_4 $   & $ = 4 $
                                \end{tabular};
                        \end{dcases}
                \end{math}

                \paragraph{}
                \begin{math}
                        A = \begin{bmatrix}
                                1 & 0 &  2 & -2 \\
                                3 & 1 & -1 &  1
                        \end{bmatrix};
                        \hfill
                        b = \begin{bmatrix}
                                2 \\
                                4
                        \end{bmatrix};
                        \hfill
                        c = \begin{bmatrix}
                                -5 \\
                                -1 \\
                                -1 \\
                                1
                        \end{bmatrix};
                        \hfill
                \end{math}

                \paragraph{}
                \begin{math}
                        B = \begin{bmatrix}
                                A_2 & A_3
                        \end{bmatrix}
                        = \begin{bmatrix}
                                0 & 2 \\
                                1 & -1
                        \end{bmatrix};
                \end{math}
                \quad
                \begin{math}
                        \begin{dcases}
                                \begin{aligned}
                                        B\left(1\right) & = 2; \\
                                        B\left(2\right) & = 3;
                                \end{aligned}
                        \end{dcases}
                \end{math}

                \paragraph{}
                \begin{math}
                        \framebox{
                                $
                                B^{-1} = \frac{1}{det(B)} *
                                \begin{bmatrix}
                                        Cof(B)
                                \end{bmatrix}^T
                                $
                        } =
                        - \frac{1}{2} \times \begin{bmatrix}
                                -1 & -1 \\
                                -2 & 0
                        \end{bmatrix} =
                        \begin{bmatrix}
                                1/2 & 1/2 \\
                                1 & 0
                        \end{bmatrix};
                \end{math}


                \paragraph{}
                \begin{math}
                        \framebox{ $ u^T = c^{T}_{B}B^{-1} $ } = \begin{bmatrix}
                                -1 & -1
                        \end{bmatrix} \begin{bmatrix}
                                1/2 & 1/2 \\
                                1 & 0
                        \end{bmatrix} = \begin{bmatrix}
                                ( -\frac{1}{2} -1 ) & -\frac{1}{2}
                        \end{bmatrix} = \begin{bmatrix}
                                -\frac{3}{2} & -\frac{1}{2}
                        \end{bmatrix};
                \end{math}

                \paragraph{}
                \begin{math}
                        \framebox{ $ \overline{c}_F = c_j - u^{T}A_j $ } = \begin{dcases}
                                \begin{tabular}{r r l l l}
                                        $ \overline{c_1} = $ & $ -5 $ & $ - \begin{bmatrix}
                                                -\frac{3}{2} & -\frac{1}{2}
                                        \end{bmatrix} \begin{bmatrix}
                                                1 \\
                                                3
                                        \end{bmatrix} $ & $ = -5 - (-\frac{3}{2} - \frac{1}{2}) $ & $ = \framebox{ -2 } $ \\
                                        $ \overline{c_4} = $ & $ 1 $ & $ - \begin{bmatrix}
                                                -\frac{3}{2} & -\frac{1}{2}
                                        \end{bmatrix} \begin{bmatrix}
                                                -2 \\
                                                1
                                        \end{bmatrix} $ & $ = 1 - (-3 - \frac{1}{2}) $ & $ = \frac{9}{2} $
                                \end{tabular}
                        \end{dcases};
                \end{math}

                \paragraph{}
                \begin{math}
                        B = \begin{bmatrix}
                                A_2 & A_2
                        \end{bmatrix}
                \end{math} non \`{e} ottima per $ \overline{c_F} < 0 $.

                \paragraph{}
                La variabile entrante per $ \overline{c_F} < 0$ \quad \`{e} \quad \framebox{ $ \overline{c_h} = \overline{c_1} $ } $ = -2 $ \quad { \large $\implies$ }\quad \framebox{ $ x_h = x_1 $ }.

                \paragraph{}
                \begin{math}
                        \framebox{ $ \overline{b} = B^{-1}b $ } = \begin{bmatrix}
                                1/2 & 1/2 \\
                                1 & 0
                        \end{bmatrix} \begin{bmatrix}
                                2 \\
                                4
                        \end{bmatrix} = \begin{bmatrix}
                                3 \\
                                2
                        \end{bmatrix};
                \end{math}

                \paragraph{}
                \begin{math}
                        \framebox{ $ \overline{A}_h = \overline{A}_1 = B^{-1}A_1 $ } = \begin{bmatrix}
                                1/2 & 1/2 \\
                                1 & 0
                        \end{bmatrix} \begin{bmatrix}
                                1 \\
                                3
                        \end{bmatrix} = \begin{bmatrix}
                                1 \\
                                1
                        \end{bmatrix};
                \end{math}

                \paragraph{}
                \begin{math}
                        \parbox{0.45 \linewidth}{
                                \begin{math}
                                        \frac{\overline{b}_m}{\overline{a}_{m,h}} = \begin{dcases}
                                                \begin{aligned}
                                                        \frac{\overline{b}_1}{\overline{a}_{1,h}} = & \frac{\overline{b}_1}{\overline{a}_{1,1}} = \frac{3}{1} = 3 \\
                                                        \frac{\overline{b}_2}{\overline{a}_{2,h}} = & \frac{\overline{b}_2}{\overline{a}_{2,1}} = \frac{2}{1} = \framebox{ 2 }
                                                \end{aligned}
                                        \end{dcases};
                                \end{math}
                        }
                        \parbox{0.25 \linewidth}{
                                \framebox{ $ \frac{\overline{b}_m}{\overline{a}_{m,h}} > 0 $ }
                                \newline
                                { \scriptsize problema non illimitato. }
                        }
                        \parbox{0.3 \linewidth}{
                                $ x_{B(2)} = x_3 $ \newline { \scriptsize \`{e} la variabile uscente. }
                        }
                \end{math}

        \end{preview}
\end{document}
