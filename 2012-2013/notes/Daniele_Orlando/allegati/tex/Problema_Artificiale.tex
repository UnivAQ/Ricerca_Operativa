\documentclass{article}

\usepackage[active,tightpage]{preview}
\setlength\PreviewBorder{2pt}
\usepackage[paperwidth=300pt]{geometry}

\usepackage{rotating}
\usepackage{color}
\usepackage[usenames,dvipsnames,table]{xcolor}
\usepackage{mathtools}
\usepackage{breqn}
\usepackage{setspace}

\newcommand{\specialcell}[2][c]{%
        \begin{tabular}[#1]{
                @{}c@{}}#2
        \end{tabular}
}

\begin{document}
        \begin{preview}
                \paragraph{}
                \begin{equation*}
                        min \; 3 x_1 + 4 x_2 + 6 x_3
                \end{equation*}
                \begin{math}
                        \begin{dcases}
                                \begin{tabular}{r r r l}
                                        $ x_1 $ & $ + 3 x_2 $ & $ + 4 x_3 $ & $ = 1 $ \\
                                        $ 2 x_1 $ & $ + x_2 $ & $ + 3 x_3 $ & $ = 2 $
                                \end{tabular} \\
                                \quad
                                x_1, x_2, x_3 \geq{0}
                        \end{dcases};
                \end{math}
                \begin{equation*}
                        min \; y_1 + y_2
                \end{equation*}
                \begin{math}
                        \begin{dcases}
                                \begin{tabular}{r r r r r l}
                                        $ x_1 $ & $ + 3 x_2 $ & $ + 4 x_3 $ & $ + y_1 $ &  & $ = 1 $ \\
                                        $ 2 x_1 $ & $ + x_2 $ & $ + 3 x_3 $ & & $ + y_2 $ & $ = 2 $
                                \end{tabular} \\
                                \quad
                                x_1, x_2, x_3, y_1, y_2 \geq{0}
                        \end{dcases};
                \end{math}

                \paragraph{}
                \begin{tabular}{r | r r r r r | r | r}
                        \cline{2-7}
                        & 0 & 0 & 0 & 1 & 1 & 0 & \\
                        \cline{2-7}
                        & 1 & 3 & 4 & 1 & 0 & 1 & \textcolor{blue}{ $ y_1 $ } \\
                        & 2 & 1 & 3 & 0 & 1 & 2 & \textcolor{blue}{$ y_2 $ } \\
                        \cline{2-7}
                \end{tabular}

                \paragraph{}
                \begin{tabular}{r | r r r r r | r | r}
                        \cline{2-7}
                        & -3 & -4 & -7 & 0 & 0 & -3 & \\
                        \cline{2-7}
                        & 1 & 3 & 4 & 1 & 0 & 1 & \textcolor{blue}{ $ y_1 $ } \\
                        & 2 & 1 & 3 & 0 & 1 & 2 & \textcolor{blue}{$ y_2 $ } \\
                        \cline{2-7} \multicolumn{6}{r}{
                                {\fontsize{6pt}{4pt}\selectfont
                                        \specialcell{
                                                Forma Canonica \\
                                                \textcolor{gray}{
                                                        Riga \#0 - \#1 - \#2
                                                }
                                        }
                                }
                        }
                \end{tabular}

                \paragraph{}
                \begin{tabular}{r | r r r r r | r | r}
                        \cline{2-7}
                        & -3 & -4 & \textcolor{blue}{-7} & 0 & 0 & -3 & \\
                        \cline{2-7}
                        & 1 & 3 & \textcolor{red}{4} & 1 & 0 & 1 & \textcolor{blue}{ $ y_1 $ } \\
                        & 2 & 1 & 3 & 0 & 1 & 2 & \textcolor{blue}{$ y_2 $ } \\
                        \cline{2-7}
                \end{tabular}

                \paragraph{}
                \begin{tabular}{r | r r r r r | r | r}
                        \cline{2-7}
                        & \textcolor{blue}{-5/4} & -9/4 & 0 & 7/4 & 0 & -5/4 & \\
                        \cline{2-7}
                        & 1/4 & 3/4 & 1 & 1/4 & 0 & 1/4 & \textcolor{blue}{ $ x_3 $ } \\
                        & \textcolor{red}{5/4} & -5/4 & 0 & -3/4 & 1 & 5/4 & \textcolor{blue}{$ y_2 $ } \\
                        \cline{2-7}
                \end{tabular}

                \paragraph{}
                \begin{tabular}{r | r r r r r | r | r}
                        \cline{2-7}
                        & 0 & 0 & 0 & 1 & 1 & 0 & \\
                        \cline{2-7}
                        & 0 & 1 & 1 & 2/5 & -1/4 & 0 & \textcolor{blue}{ $ x_3 $ } \\
                        & 1 & -1 & 0 & -3/5 & 4/5 & 1 & \textcolor{blue}{$ x_1 $ } \\
                        \cline{2-7}
                \end{tabular}

                \begin{center}
                        \parbox{0.9 \linewidth} {
                                {\scriptsize
                                        \hfill\newline
                                        Soluzione ottima
                                        { \setlength{\arraycolsep}{3pt} $ x^T = \begin{bmatrix} 1 & 0 & 0 & 0 & 0 \end{bmatrix} $ } di valore 0. \newline
                                        Se fosse rimasta in base una $ y $ avremmo eseguito un'operazione di pivot sulla riga corrispondente alla $ y $ facendo entrare una qualsiasi $ x $. \newline
                                        \hfill\newline
                                        Eliminiamo le colonne fuori base $ y_1 $ e $ y_2 $ \newline
                                        e ripristiniamo la f.o. originaria.
                                }
                        }
                \end{center}

                \paragraph{}
                \begin{tabular}{r | r r r | r | r}
                        \multicolumn{1}{r}{} & \textcolor{red}{ $ x_1 $ } & & \multicolumn{1}{r}{\textcolor{red}{ $ x_3 $}}  \\
                        \cline{2-5}
                        & \textcolor{red}{3} & 4 & \textcolor{red}{6} & 0 & \\
                        \cline{2-5}
                        & 0 & 1 & 1 & 0 & \textcolor{red}{ $ x_3 $ } \\
                        & 1 & -1 & 0 & 1 & \textcolor{red}{ $ x_1 $ } \\
                        \cline{2-5}
                \end{tabular}

                \paragraph{}
                \begin{tabular}{r | r r r | r | r}
                        \cline{2-5}
                        & 0 & 1 & 0 & -3 & \\
                        \cline{2-5}
                        & 0 & 1 & 1 & 0 & \textcolor{blue}{ $ x_3 $ } \\
                        & 1 & -1 & 0 & 1 & \textcolor{blue}{$ x_1 $ } \\
                        \cline{2-5} \multicolumn{6}{r}{
                                {\fontsize{6pt}{4pt}\selectfont
                                        \specialcell{
                                                Forma Canonica \\
                                                \textcolor{gray}{
                                                        \#0 - ( 6 x \#1 ) - ( 3 x \#2 )
                                                }
                                        }
                                }
                        }
                \end{tabular}

                \begin{center}
                        \parbox{0.9 \linewidth} {
                                {\scriptsize
                                        \hfill\newline
                                        Eseguiamo la Fase II del Simplesso. \newline
                                        Niente da eseguire, la base \`{e} ottima.
                                }
                        }
                \end{center}
        \end{preview}
\end{document}
